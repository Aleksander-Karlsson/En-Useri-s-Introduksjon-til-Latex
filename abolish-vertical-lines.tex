\documentclass{article}
\usepackage[utf8]{inputenc}
\usepackage{booktabs}
\usepackage{float}
\usepackage{caption}
\usepackage{makecell}

\title{The moral case for abolishing vertical lines}
\author{Aleksander Karlsson}
\date{March 2021}

\begin{document}
\maketitle


\noindent Eks tabell som ser helt forferdelig ut men som viser de mest brukte funksjonene i tabeller:
\begin{table}[h!]
    \centering
    \caption{tekst som forklarer tabellen}
    \label{tab:my_label}
    \vspace{2mm}
    \begin{tabular}{|rc|l||r} % Setter stilen på kolonnene, "c" er centered, "l" og "r" er right/left adjusted, "|" gir en strek mellom, men det er fint mulig å ikke ha strek, eller f.eks dobbeltstrek
        \hline
        \textbf{Lorem} & \textbf{ipsum} & \textbf{dolor} & \textbf{sit} \\ \hline
        amet,          & consectetur    & adipiscing     & elit, \\ 
        sed            & do             & eiusmod        & tempor \\ \hline\hline %\hline setter linje mellom radene
        incididunt     & ut             & labore         & et \\
    \end{tabular}
\end{table}



\begin{table}[H]
    \caption{Mapping of Learning Objectives to Game Activity as in the table in} % TODO: cite Serious Gaming book table 6.1 og page 91
    \label{tab:learningobjectives}
    % \vspace{1mm}
    \centering
    \addtolength{\leftskip}{-12cm}
    \addtolength{\rightskip}{-12cm}
    \begin{tabular}{lll}
        \toprule
        \textbf{Learning objectives}  & \textbf{Learning activities} & \textbf{Game activities}\\\midrule%\Xhline{2\arrayrulewidth} %%\hline
        
        \makecell[l]{
        Understand how the\\
        value of users’ data\\ 
        changes with its amount} & 
        \makecell[l]{
        Gather data based on\\
        different techniques from\\
        the perspective of a \\
        corporation} & 
        \makecell[l]{
        Create subsidiaries that\\
        generate data through legal \\
        or illegal means. E.g. search-\\
        motor, social network, hacking}\\\midrule 
        
        \makecell[l]{
        Understand how different \\
        types of users’ data can be\\
        merged together for more value} & 
        \makecell[l]{
        Combine and trade your data\\
        from multiple sources to\\
        increase its value} & 
        \makecell[l]{
        Net-worth of your company\\
        changes with the variety \\ 
        and scarcity of the data gathered} \\\bottomrule
    \end{tabular}
\end{table}
% VS
\begin{table}[H]
    
    \caption{Mapping of Learning Objectives to Game Activity as in the table in} % TODO: cite Serious Gaming book table 6.1 og page 91
    \label{tab:learningobjectives2}
    % \vspace{1mm}
    \centering
    \addtolength{\leftskip}{-12cm}
    \addtolength{\rightskip}{-12cm}
    \begin{tabular}{|l|l|l|}
        \hline
        \textbf{Learning objectives}  & \textbf{Learning activities} & \textbf{Game activities}\\\hline%\Xhline{2\arrayrulewidth} %%\hline
        
        \makecell[l]{
        Understand how the\\
        value of users’ data\\ 
        changes with its amount} & 
        \makecell[l]{
        Gather data based on\\
        different techniques from\\
        the perspective of a \\
        corporation} & 
        \makecell[l]{
        Create subsidiaries that\\
        generate data through legal \\
        or illegal means. E.g. search-\\
        motor, social network, hacking}\\\hline
        
        \makecell[l]{
        Understand how different \\
        types of users’ data can be\\
        merged together for more value} & 
        \makecell[l]{
        Combine and trade your data\\
        from multiple sources to\\
        increase its value} & 
        \makecell[l]{
        Net-worth of your company\\
        changes with the variety \\ 
        and scarcity of the data gathered} \\\hline
    \end{tabular}
\end{table}

\begin{table}[H]
    \caption{Overview of the concepts of the game}
    \label{tab:overview}
    \centering
    \begin{tabular}{ll}
        \toprule
        \textbf{Title}  & 
        \makecell[l]{
            Data Hoarders
        }\\\midrule
        \textbf{Theme}  & 
        \makecell[l]{
            Data aggregation
        }\\\midrule
    
        \textbf{Target group}  & 
        \makecell[l]{
            16+ teenager to adults\\ 
            It is a complex idea
        }\\\midrule% 18+ for strip club expansion
        
        \textbf{Genre}  & 
        \makecell[l]{
            Turn-based Strategic Game\\ 
            Because it allows for weighing and thinking
        }\\\midrule% Turn-based does not require you to be mechanically inclined, turn-timer to encourage risky behaviour (XCOM quote) 
        
        % Has elements of roleplaying and allows for roleplaying, since the player will take the role of the "bad guy" to obtain data and thus get a better understanding of value of data. 
        
        \textbf{Setting}  & 
        \makecell[l]{
            Real world\\
            In order to analyse the risk taking it has\\
            to be as close to the real world as possible
        }\\\midrule
        
        \textbf{Constraints}  & 
        \makecell[l]{
            Turns have a time limit, and players can\\
            adjust it before starting the game\\
            Number of players: 2-8\\
            Limited number of companies in each region in game 
        }\\\midrule
    
        \textbf{Target platform}  & 
        \makecell[l]{
            Personal Computer\\
            In such complex games the use of mouse and keyboard\\
            will allow for presenting a lot of information
        }\\\bottomrule
    \end{tabular}
\end{table}
%VS
\begin{table}[H]
    \caption{Overview of the concepts of the game}
    \label{tab:overview2}
    \centering
    \begin{tabular}{|l|l|}
        \hline
        \textbf{Title}  & 
        \makecell[l]{
            Data Hoarders
        }\\\hline
        \textbf{Theme}  & 
        \makecell[l]{
            Data aggregation
        }\\\hline
    
        \textbf{Target group}  & 
        \makecell[l]{
            16+ teenager to adults\\ 
            It is a complex idea
        }\\\hline% 18+ for strip club expansion
        
        \textbf{Genre}  & 
        \makecell[l]{
            Turn-based Strategic Game\\ 
            Because it allows for weighing and thinking
        }\\\hline% Turn-based does not require you to be mechanically inclined, turn-timer to encourage risky behaviour (XCOM quote) 
        
        % Has elements of roleplaying and allows for roleplaying, since the player will take the role of the "bad guy" to obtain data and thus get a better understanding of value of data. 
        
        \textbf{Setting}  & 
        \makecell[l]{
            Real world\\
            In order to analyse the risk taking it has\\
            to be as close to the real world as possible
        }\\\hline
        
        \textbf{Constraints}  & 
        \makecell[l]{
            Turns have a time limit, and players can\\
            adjust it before starting the game\\
            Number of players: 2-8\\
            Limited number of companies in each region in game 
        }\\\hline
    
        \textbf{Target platform}  & 
        \makecell[l]{
            Personal Computer\\
            In such complex games the use of mouse and keyboard\\
            will allow for presenting a lot of information
        }\\\hline
    \end{tabular}
\end{table}

\begin{table}[H]
    \caption{I tabeller skal denne ``bildeteksten'' være \textbf{OVER} tabellen!}
    \label{tab:keybinds}
    \centering
    \begin{tabular}{cc}
        Key & Description \\
        \hline\hline
        W & Move inwards \\
        S & Move outwards \\
        A & Move to the left \\
        D & Move to the right \\\hline
    \end{tabular}
\end{table}
% VS
\begin{table}[H]
    \caption{I tabeller skal denne ``bildeteksten'' være \textbf{OVER} tabellen!}
    \label{tab:keybinds2}
    \centering
    \begin{tabular}{|c|c|}
        \hline
        Key & Description \\\hline
        W & Move inwards \\\hline
        S & Move outwards \\\hline
        A & Move to the left \\\hline
        D & Move to the right \\\hline
    \end{tabular}
\end{table}


\begin{table}[H]
    \caption{Tools we have chosen to support digital teamwork}
    \label{tab:cooptools}
    \vspace{1mm}
    \centering
    \begin{tabular}{|l|l|l|}
        \hline
        \#  & \textbf{Tool} & \textbf{Affordances}\\ \Xhline{2\arrayrulewidth} %\hline
        1   & Miro          & \makecell[l]{Simultaneity, reviewability, revisability, \\enables creativity, interactive}\\\hline
        2   & Discord       & Face2Face, audibility, visibility\\\hline
        3   & Overleaf      & Simultaneity, reviewability, revisability, interactive\\\hline
    \end{tabular}
\end{table}
% VS
\begin{table}[H]
    \caption{Tools we have chosen to support digital teamwork}
    \label{tab:cooptools2}
    \vspace{1mm}
    \centering
    \begin{tabular}{lll}
        \#  & \textbf{Tool} & \textbf{Affordances}\\\toprule %\Xhline{2\arrayrulewidth} %\hline
        1   & Miro          & \makecell[l]{Simultaneity, reviewability, revisability, \\enables creativity, interactive}    \\ \hline
        2   & Discord       & Face2Face, audibility, visibility             \\ \hline
        3   & Overleaf      & Simultaneity, reviewability, revisability, interactive    \\ 
    \end{tabular}
\end{table}




\end{document}